\documentclass[a4]{article}
\usepackage{rsllisting}
\lstset{language=rsl}

%Comments in the Latex source are written after the % sign

\title{Course 02263 Mandatory Assignment 1, 2018} 

\author{name1 (study number1),\\ 
        name2 (study number2)}


\begin{document}

\maketitle

\tableofcontents
\newpage

\section{Introduction}
This document contains a solution to ... . 

\section{Specification of Types and Auxiliary Functions}


\lstinputlisting{Basics.rsl}  

{\em Here you must informally explain the purpose of your auxiliary functions}

\section{Requirement Specification}

\lstinputlisting{Requirements.rsl}  

{\em Here you must informally explain which requirements the post
  condition expresses.}

\section{Refined, explicit specification}

\lstinputlisting{Design.rsl}  

{\em Here you must  explain the idea behind your algorithm.}
{\em If you in this scheme introduce auxiliary functions in order to explicitly
  define the plan function, you must explain these as well.}

\section{Testing by translation to SML}

\subsection{Test specification}

\lstinputlisting{DesignTest.rsl}
        
{\em Here you must  explain what the purpose of the individual test
  cases are.}


\subsection{Results of evaluations}


The results of evaluating the SML translation of the RSL test cases are:

{\em Here you must  insert the output of the test run.}



\noindent From this we can conclude ...


\end{document}
