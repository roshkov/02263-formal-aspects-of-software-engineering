\documentclass[a4]{article}
\usepackage{rsllisting}
\lstset{language=rsl}

%Comments in the Latex source are written after the % sign

\title{Course 02263 Mandatory Assignment 1, 2018} 

\author{Marc Storm Larsen (s144452),\\ 
        Mario Padilla Carrion (s172387)}


\begin{document}

\maketitle

\tableofcontents
\newpage

\section{Introduction}
This document contains a solution to ... . 

\section{Specification of Types and Auxiliary Functions}


\lstinputlisting{Basics.rsl}  

% Informal explanation of the purpose of auxiliary functions.
\begin{description}
  \item[areRelatives] \hfill \\ This function can be used to test whether two persons belong to the same family.
\end{description}

\section{Requirement Specification}

\lstinputlisting{Requirements.rsl}  

% Informal explanation of which requirements the post condition expresses.
\begin{description}
  \item[isCorrectPlan] \hfill \\ This function can be used to test if all persons of all families have been assigned to a table. As well as no more than one person from the same family is assigned to the same table. Furthermore that no tables are empty.
  \item[isCorrectTable] \hfill \\ This function can be used to test if all persons assigned to the same table are from different families.
  \item[isWellformed] \hfill \\ Families is well formed if every person is in a family and only one family, and that a family contains at least one person.
\end{description}

\section{Refined, explicit specification}

\lstinputlisting{Design.rsl}  

{\em Here you must  explain the idea behind your algorithm.}
{\em If you in this scheme introduce auxiliary functions in order to explicitly
  define the plan function, you must explain these as well.}

\section{Testing by translation to SML}

\subsection{Test specification}

\lstinputlisting{DesignTest.rsl}
        
{\em Here you must  explain what the purpose of the individual test
  cases are.}


\subsection{Results of evaluations}


The results of evaluating the SML translation of the RSL test cases are:

{\em Here you must  insert the output of the test run.}



\noindent From this we can conclude ...


\end{document}
