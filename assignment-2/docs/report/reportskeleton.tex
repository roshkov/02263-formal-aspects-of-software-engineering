\documentclass[a4]{article}
\usepackage[total={6.7in,10.2in}, % text area width and height
		    top=1in, left=1in, right=1in, bottom=1in,% top and left margin
		    includefoot]{geometry}
\usepackage{rsllisting}
\lstset{language=rsl}

\title{Course 02263 Mandatory Assignment 2, 2018} 

\author{Marc Storm Larsen (s144452),\\ 
        Mario Padilla Carrion (s172387)}

\begin{document}

\maketitle

\tableofcontents
\newpage

\section{Work distribution}

\section{Introduction}
This document contains a solution to the problem of representing tram nets and tram time tables with data types, furthermore will the document describe a number of functions that can be used to check whether concrete values of the data types to describe the net and time table satisfies a number of requirements, making them wellformed.

\section{Formal Specification of Nets}

\lstinputlisting{assignment-2/src/NET.rsl}

\noindent Below are the requirements for a \emph{net} defined. These are the requirement that a \emph{net} has to fulfil in order to be considered wellformed. Each requirement will indicate what function is used to ensure the requirement. \\

\begin{description}
    \item[isWellformed] \hfill
        \begin{itemize}
            \item The \emph{net} must contain at least two \emph{stops}. Function: \verb=hasCorrectSize=.
            \item Any \emph{stop} must have a direct or indirect connection to every other \emph{stop} in the \emph{net}, i.e. a tram can travel to any \emph{stop} from any \emph{stop}. Function: \verb=isNetConnected=.
            \item A \emph{stop} can not have a direct \emph{connection} to itself. Function: \verb=noDirectSelfConnections=.
            \item A \emph{stop} can not have more than one direct \emph{connection} to the same \emph{stop}. Function: \verb=hasOneDirectConnection=.
            \item A \emph{connection} between two \emph{stops} consists of two tracks, one track for each direction, i.e. a tram can travel back and forth between the two \emph{stops} using the same \emph{connection}. Function: \verb=hasBiDirectConnection=.
        \end{itemize}
\end{description}

\noindent In the list below one will find a description of all the functions.\\

\begin{description}
    \item[hasCorrectSize] \hfill \\ Can be used to check whether or not a \emph{net} has a correct size, i.e. the number of \emph{stops} in the \emph{net}.
    \item[isNetConnected] Can be used to check that the \emph{net} is connected, i.e. there is a direct or indirect connection from any stop to all other stops, by using \verb=areConnected=.
    \item[noDirectSelfConnections] \hfill \\ Can be used to check that any \emph{stop} in the \emph{net} does not have a direct connection to itself, by using \verb=areDirectlyConnected=.
    \item[hasOneDirectConnection] \hfill \\ Can be used to check if any \emph{stop} in the \emph{net} only has one direct connection to another \emph{stop}, by using \verb=Connections=.
    \item[hasBiDirectConnection] \hfill \\ Can be used to check that all direct connections between two \emph{stops} in the \emph{net} are connecting the stops in both directions, by using \verb=areDirectlyConnected=.
    \item[isNetConnected] Can be used to check that the \emph{net} is connected, i.e. there is a direct or indirect connection from any stop to all other stops, by using \verb=areConnected=.
    \item[isIn] \hfill \\ Can be used to check whether or not a \emph{stop} is in a \emph{net}.
    \item[capacity - stop] \hfill \\ Can be used to find out what the \emph{capacity} of a \emph{stop} is.
    \item[areDirectlyConnected] \hfill \\ Can be used to check whether or not two \emph{stops} are directly connected in a \emph{net}, by using the \verb=hasDirectConnection=.
    \item[minHeadway] \hfill \\ Can be used to find out what the minimum \emph{headway} between two connected \emph{stops} is, by using the \verb=getConnection=.
    \item[capacity - connection] \hfill \\ Can be used to find out what the capacity of a connection is, by using the \verb=getConnection=.
    \item[minDrivingTime] \hfill \\ Can be used to find out what the minimum driving time between two connected \emph{stops} is, by using the \verb=getConnection=.
    % Our own observer functions
    \item[filterConnections] \hfill \\ Can be used to find a single \emph{connection} in a set of connections.
    \item[getConnection] \hfill \\ Can be used to get the \emph{connection} between two \emph{stops}, by using \verb=filterConnections=.
    \item[hasDirectConnection] \hfill \\ Can be used to check if two \emph{stops} are directly connected in one direction in a \emph{net}.
    \item[areConnected] Can be used to check if two stops are connected, either with a direct or indirect connection.
\end{description}  

\section{Formal Specification of Time Tables}

\lstinputlisting{assignment-2/src/TIMETABLE.rsl}

\begin{description}
    \item[isIn] \hfill \\ Can be used to check whether or not a tram with a given name exists in a given time table.
    \item[isWellformed] \hfill \\ 
        \begin{itemize}
            \item item 1
        \end{itemize}
\end{description}

\section{Testing the Specifications}

\subsection{Nets}

\lstinputlisting{assignment-2/src/NETTest.rsl}

\subsection{Time Tables}

\lstinputlisting{assignment-2/src/TIMETABLETest.rsl}

\section{Testing by translation to SML}

\subsection{Test specification}

\subsection{Results of evaluations}

\end{document}
