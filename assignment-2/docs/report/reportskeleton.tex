\documentclass[a4]{article}
\usepackage[total={6.7in,10.2in}, % text area width and height
		    top=1in, left=1in, right=1in, bottom=1in,% top and left margin
		    includefoot]{geometry}
\usepackage{rsllisting}
\lstset{language=rsl}

\title{Course 02263 Mandatory Assignment 2, 2018} 

\author{Marc Storm Larsen (s144452),\\ 
        Mario Padilla Carrion (s172387)}

\begin{document}

\maketitle

\tableofcontents
\newpage

\section{Work distribution}

\section{Introduction}

\section{Formal Specification of Nets}

\lstinputlisting{assignment-2/src/NET.rsl}


% ---------- NEEDS TO BE CLARIFIED ----------
% - Should we make the functions below robust, check for legal input and so on, or should it allow illegal networks to be created?
\begin{description}
    \item[isIn] \hfill \\ Can be used to check whether or not a stop is in a network.
    \item[getConnection] \hfill \\ Can be used to get the \emph{connection} between two stops.
    \item[capacity - stop] \hfill \\ Can be used to find out what the capacity of a stop is.
    \item[areDirectlyConnected] \hfill \\ Can be used to check whether or not two stops are directly connected in a network, by using the \verb=getConnection=.
    \item[minHeadway] \hfill \\ Can be used to find out what the minimum headway between two connected stops is, by using the \verb=getConnection=.
    \item[capacity - connection] \hfill \\ Can be used to find out what the capacity of a connection is, by using the \verb=getConnection=.
    \item[minDrivingTime] \hfill \\ Can be used to find out what the minimum driving time between two connected stops is, by using the \verb=getConnection=.
    \item[isWellformed] \hfill \\ 
        \begin{itemize}
            \item The network must contain at least two \emph{stops}.
            \item A \emph{stop} must have at least one \emph{connection} to another \emph{stop}.
            \item A \emph{stop} can not have a direct \emph{connection} to itself, by using the \verb=areDirectlyConnected=.
            \item A \emph{stop} can not have more than one direct \emph{connection} to another \emph{stop}.
            \item A \emph{connection} between two \emph{stops} consists of two tracks, one track for each direction, i.e. a tram can travel back and forth between the two \emph{stops} using the same \emph{connection}.
            \item Any \emph{stop} should have a direct or indirect connection to every other \emph{stop} in the network, i.e. a tram can travel to any \emph{stop} from any \emph{stop}.
            % TODO: Can we make checks that the capacity of a stop and connected is exceeded?
            % TODO: Should we make a check to see if all stops are in the network? We don't have a structure for all the StopId created, as we did in the previous assignment, where we could check that all persons from all families were part of the plan.
        \end{itemize}
\end{description}

\section{Formal Specification of Time Tables}

\lstinputlisting{assignment-2/src/TIMETABLE.rsl}

\begin{description}
    \item[isIn] \hfill \\ Can be used to check whether or not a tram with a given name exists in a given time table.
    \item[isWellformed] \hfill \\ 
        \begin{itemize}
            \item item 1
        \end{itemize}
\end{description}

\section{Testing the Specifications}

\subsection{Nets}

\lstinputlisting{assignment-2/src/NETTest.rsl}

\subsection{Time Tables}

\lstinputlisting{assignment-2/src/TIMETABLETest.rsl}

\section{Testing by translation to SML}

\subsection{Test specification}

\subsection{Results of evaluations}

\end{document}
